\usepackage[T1,T2A]{fontenc}
\usepackage[utf8]{inputenc}
\usepackage[english,russian]{babel}
\usepackage{microtype}
\usepackage{csquotes}
\usepackage{amsmath}
\usepackage{amsthm}
\usepackage{amssymb}
\usepackage{mathtext}
\usepackage{physics}
\usepackage{newfloat}
\usepackage{caption}
\usepackage{indentfirst}
\usepackage{titlesec,titletoc}
\usepackage{geometry}
\usepackage{hyperref}
\usepackage{mdframed}
\usepackage[inline]{enumitem}
\usepackage{graphicx}
\usepackage{subfig}
\usepackage{longtable}

\newcommand{\onlyinsubfile}[1]{#1}
\newcommand{\notinsubfile}[1]{}
\newcommand{\makedocroot}[0]{
    \renewcommand{\onlyinsubfile}[1]{}
    \renewcommand{\notinsubfile}[1]{##1}
}

\DeclareGraphicsExtensions{.pdf,.png,.jpg,.PNG}
\graphicspath{{./img/}{\docroot/img/}}
\captionsetup[figure]{justification=centering}
\renewcommand{\thesubfigure}{\asbuk{subfigure}}
\DeclareCaptionLabelSeparator{dotseparator}{. }
\captionsetup{labelsep=dotseparator}

\usepackage[%
    style=gost-numeric,
    citestyle=gost-numeric,
    isbn=true,
    url=true,
    defernumbers=true,
    %sorting=nyt, % sort by name, year, title
    sorting=none, % order of appearance
    bibencoding=utf8,
    backend=biber,
    language=auto, % get main language from babel
    autolang=other, % get item language from bib entry
]{biblatex}

\defbibheading{lessonbib}{\subsection{Список литературы к занятию}}
